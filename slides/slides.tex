% vim: set tabstop=2 softtabstop=2 shiftwidth=2 textwidth=80: %

\documentclass[10pt, compress, xcolor={usenames,dvipsnames}]{beamer}
\usepackage[utf8x]{inputenc}

%%% Theme and style
\usetheme{m}

\renewcommand{\emph}[1]{\alert{#1}}


%%% Packages %%%

% Use T1 and a modern font family for better support of accents, etc.
\usepackage[T1]{fontenc}
\usepackage{palatino}  % Palatino

% Language support
\usepackage[english]{babel}

% Support for easily changing the enumerator in
% enumerate-environments.
\usepackage{enumerate}

% Support for importing images
%\usepackage{graphicx}

% Use hyperlinks
\usepackage{hyperref}

% Don't load xcolors package in beamer: use document class option
% instead...
%\usepackage[usenames,dvipsnames]{xcolor}

% Use colors in tables
%\usepackage[pdftex]{colortbl}

% A nice monospace font for listings, etc.
\usepackage[scaled]{beramono}
%\usepackage{inconsolata}

% Using TikZ for diagrams
\usepackage{tikz}
\usetikzlibrary{arrows,fit,matrix,positioning,decorations.pathreplacing}
\usepackage{tikz-cd} % for CM-arrow tips.

% Don't use externalize with gradients!!!
%\usetikzlibrary{external,arrows,fit,matrix,positioning}
%\tikzexternalize % Activate externalizing TikZ graphics.


%%% JSON LISTING

\usepackage{listings}

\colorlet{JSONPUNCTCOLOR}{red!60!black}
\definecolor{JSONBACKGROUNDCOLOR}{HTML}{EEEEEE}
\definecolor{JSONDELIMCOLOR}{RGB}{20,105,176}
\colorlet{JSONNUMBERCOLOR}{magenta!60!black}

\lstdefinelanguage{json}{
    captionpos=b,
    basicstyle=\normalfont\ttfamily\small,
    %numbers=left,
    %numberstyle=\scriptsize,
    %stepnumber=1,
    %numbersep=8pt,
    showstringspaces=false,
    breaklines=true,
    frame=single,
    backgroundcolor=\color{JSONBACKGROUNDCOLOR},
    literate=
     *{0}{{{\color{JSONNUMBERCOLOR}0}}}{1}
      {1}{{{\color{JSONNUMBERCOLOR}1}}}{1}
      {2}{{{\color{JSONNUMBERCOLOR}2}}}{1}
      {3}{{{\color{JSONNUMBERCOLOR}3}}}{1}
      {4}{{{\color{JSONNUMBERCOLOR}4}}}{1}
      {5}{{{\color{JSONNUMBERCOLOR}5}}}{1}
      {6}{{{\color{JSONNUMBERCOLOR}6}}}{1}
      {7}{{{\color{JSONNUMBERCOLOR}7}}}{1}
      {8}{{{\color{JSONNUMBERCOLOR}8}}}{1}
      {9}{{{\color{JSONNUMBERCOLOR}9}}}{1}
      {:}{{{\color{JSONPUNCTCOLOR}{:}}}}{1}
      {,}{{{\color{JSONPUNCTCOLOR}{,}}}}{1}
      {\{}{{{\color{JSONDELIMCOLOR}{\{}}}}{1}
      {\}}{{{\color{JSONDELIMCOLOR}{\}}}}}{1}
      {[}{{{\color{JSONDELIMCOLOR}{[}}}}{1}
      {]}{{{\color{JSONDELIMCOLOR}{]}}}}{1},
}
\lstnewenvironment{lstjson}{\lstset{language=json}}{}

%%%% Custom macros %%%%
\newif\ifcompileTreeSlides
%\compileTreeSlidesfalse
\compileTreeSlidestrue

\newcommand{\REST}[1]{{\color{NavyBlue}\ttfamily{#1}}}


%%% Document info %%%

\title{Large-scale Information Extraction from Neuroscientific Literature}

\author{Marco Antognini}

% To show the TOC at the beginning of each section, uncomment this:
% \AtBeginSection[]
% {
%   \begin{frame}<beamer>{Outline}
%     \tableofcontents[currentsection]
%   \end{frame}
% }

% To show the TOC at the beginning of each subsection, uncomment this:
% \AtBeginSubsection[]
% {
%   \begin{frame}<beamer>{Outline}
%     \tableofcontents[currentsection,currentsubsection]
%   \end{frame}
% }


% To uncover everything in a step-wise fashion, uncomment this:
% \beamerdefaultoverlayspecification{<+->}

\setbeamertemplate{footline}[frame number]

\date{%
  \small Spring 2015\\[2em]
  \includegraphics[height=7mm]{img/epfl-logo}}


%%% Start of the actual document %%%

\begin{document}

\begin{frame}
  \titlepage
\end{frame}

% No outline, too short a talk...
% \begin{frame}{Outline}
%   \tableofcontents
%   % You might wish to add the option [pausesections]
% \end{frame}

% No sections or subsections, too short a talk...
% \section{Motivation}
% \subsection*{}

\begin{frame}[fragile]{Motivation}
  TODO
\end{frame}

\begin{frame}[fragile]{Example -- Sentences}

  \begin{exampleblock}{}
    Terminologies which lack semantic connectivity hamper the effective search
    in biomedical fact databases and document retrieval systems. We here focus
    on the integration of two such isolated resources, the term lists from the
    protein fact database UNIPROT and the indexing vocabulary MESH from the
    bibliographic database MEDLINE.
  \end{exampleblock}

\end{frame}

\begin{frame}[fragile]{Example -- Sentences}

  \REST{GET /annotate/bluima.sentence?text=<\ldots>}

  \vspace{1em}

  \begin{lstjson}
"DocumentAnnotation" : [
  { "begin" : 0,    "end" : 328,  "language" : "en" }
],
"Sentence" : [
  { "begin" : 0,    "end" : 135,
    "componentId" :
        "de.julielab.types.OpenNLPSentenceDetector" },
  { "begin" : 136,  "end" : 32
    "componentId" :
        "de.julielab.types.OpenNLPSentenceDetector" }
]
  \end{lstjson}

\end{frame}

\plain{\Huge Demo!}

\begin{frame}{Thank you!}
  \begin{center}
    \Huge Questions?
  \end{center}
  \vspace{2em}
  \begin{center}
    \emph{Check it out}\vspace{.5em}\\
    \url{http://sherlok.io}\\
  \end{center}
\end{frame}

\end{document}
